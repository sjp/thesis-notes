\documentclass[11pt, a4paper]{article}

\newcommand{\grid}{\textsf{grid}}
\newcommand{\R}{\textsf{R}}
\newcommand{\gridSVG}{\textsf{gridSVG}}
\newcommand{\JS}{\textsf{JavaScript}}
\newcommand{\pkg}[1]{\textsf{#1}}
\newcommand{\code}[1]{\texttt{#1}}
\newcommand{\tsc}[1]{{\small \textsc{#1}}}

\usepackage[margin = 30mm]{geometry}
\usepackage[usenames,dvipsnames,svgnames,table]{xcolor}
\usepackage{parskip}
%\usepackage{microtype}
% Defining the file/font/lang formats we want.
\usepackage[utf8]{inputenc}
\usepackage[OT1]{fontenc}
\usepackage[newzealand]{babel}
\usepackage{hyperref}

% Document properties.
\newcommand{\doctitle}{Dynamic and Reactive Web-based Graphics for \R{}}
\newcommand{\docauthor}{Simon J. Potter}
\newcommand{\docdate}{February 18, 2013}
\title{\doctitle{}}
\author{\docauthor{} \\ \url{simon.potter@auckland.ac.nz}}
\date{\docdate{}}
\hypersetup{pdftitle = {\doctitle{} | \docauthor{}},
            pdfauthor = {\docauthor{}},
            pdfborder = {0 0 0.5}}

\begin{document}

\maketitle

A collection of tools has been developed to create and present
web-based graphics via the popular statistical software, \R{}.

\begin{itemize}
  \item \gridSVG{} --- Export \grid{} graphics to \tsc{SVG}.
  \item \pkg{selectr} --- Translate CSS Selectors to XPath Expressions.
  \item \pkg{animaker} --- Generate Animation Sequence Descriptions.
\end{itemize}

The proliferation of \emph{reactive} web-based graphics using
libraries such as D3 (\url{http://d3js.org/}) shows that there is
significant interest in creating reactive graphics in web
browsers. These reactive graphics are simply not possible to draw in R
as R is only capable of drawing \emph{static} graphics. However, the
great strength of R is its wealth of statistical functions and
graphics, which are not present within web browsers. The collective
aim of the developed packages is to act as a bridge between R and D3
for reactive, statistical graphics to be created.

The \gridSVG{} package exports \pkg{grid} plots as \tsc{SVG}
images. It has been developed from a proof-of-concept to a complete
solution where it is capable of exporting all \grid{}
graphics. \gridSVG{} is able to generate images that are capable of
animation and interactivity (via \JS{} event handlers and
hyperlinking). However, originally it did not easily allow
manipulation of the image after it had been created. Now \pkg{gridSVG}
has been extensively modified to support the goal of generating
reactive graphics. Two major features have been added; exporting of a
\gridSVG{} coordinate system and node based \tsc{SVG} document
generation.

The node based approach to drawing \tsc{SVG} is a result of using the
\pkg{XML} package. The two main reasons for rewriting to use \pkg{XML}
are because of the ability to construct in-memory images, and to
insert, remove or modify \tsc{XML} nodes at any location within our
\tsc{SVG} document. Drawing images in memory allows us to more easily
serve \gridSVG{} images over the web as it avoids the need to touch a
hard disk for storage, as is currently the case for all graphics
devices not dependent on a GUI. The ability to insert nodes at any
location is particularly useful if we wish to modify the image, as we
can add, remove or modify an \tsc{SVG} element, and its children. The
biggest advantage of using \pkg{XML} is we can \emph{modify} an image
after it has been drawn. An example of such behaviour would be to
remove, add, or alter \tsc{SVG} content to customise the resulting
plot.

The exporting of \grid{}'s coordinate system allows us to re-use a
\grid{} viewport even after it has been exported via \gridSVG{}. In
\JS{}, this is useful because we are able to add extra graphic objects
to a plot, relative to a scale, without the need to use \R{}. The
exported coordinate information can also be imported into \R{} to
generate parts of a plot, allowing one to modify an existing plot.

Commonly in a web browser, particularly with any of the popular \JS{}
libraries, we are able to programmatically select content in a web
page using CSS3 Selectors. These are often very readable and terse
expressions that make it easy for developers to target particular
pieces in a web page. It would be ideal if we could specify parts of
an \tsc{XML} document (using the \pkg{XML} package) with the same
convenient notation. The \pkg{XML} package does provide a notation
that performs this task, called XPath. However, it is not commonly
employed within web browsers and its usage is limited to those
familiar with a web-based environment. A package has been created,
\pkg{selectr}, which translates CSS3 Selectors into XPath expressions,
allowing us to use the same notation within \R{} (with \pkg{XML}) as
we would in a web browser. Convenience functions have also been
created that mimic the \code{querySelector()} method present in a web
browser. This method retrieves matching nodes based on a CSS
selector. The creation of the \pkg{selectr} package eases manipulation
of \tsc{HTML} and \tsc{XML} documents in \R{}.

If we have an image within the browser that we wish to animate, we
first need to determine how the image is to be animated. A problem
with animation is that actions are often influenced by previous
animations, and thus need to be sequenced in a particular order. Using
a \JS{} library like D3 (\url{http://d3js.org/}), we would have to
hard-code the sequencing ourselves, but this has limitations. The
first of these limitations is that we have to keep track of the
relationships between animations ourselves. For example, consider a
sequence of animations that happens one after the other. If one of the
earlier animations requires a longer period of time, it will affect
any following animations. Without any animation management, we would
need to modify all following animations to account for this.

To manage these sequences of animations, the \pkg{animaker} package
was created. It allows us to describe in \R{} how an animation should
be arranged. The result of this description is that it allows us to
export the times and durations of each animation, along with the
context in which the animation occurred. The context gives us
information such as how many times the animation has been repeated
within a sequence. If we export this information to \tsc{JSON} via the
\pkg{RJSONIO} package, then this information is available to us in
\JS{}. We can use this information to build a sequence of animations
without being concerned about when they occur, or for how long.

The advantage of each of these packages is that they can be used
independently. Although they can be combined to produce dynamic and
reactive graphics, each package is useful in its own right. This is
particularly important because we can produce these graphics
regardless of the underlying mechanism used to send information to and
from \R{}. Therefore we could use packages such as \pkg{websockets},
\pkg{shiny}, \pkg{Rook}, etc. without being limited in the type of
graphics we can produce. This means that by combining the browser with
\R{}, we are able to produce interactive graphics that can
\emph{react} to changes in the browser. As a result animation can be
applied to these changes, clearly demonstrating the visual effects of
a change in state. The development of these packages allows us to be
able to produce \emph{reactive} graphics that are both interactive and
able to be animated.

\end{document}
